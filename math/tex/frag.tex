\documentclass[11pt,reqno]{amsart}
\usepackage[document]{ragged2e}
\usepackage{tgadventor}%font
\usepackage{fullpage}% sets page margins
\usepackage{amsthm}%theorem environments
\usepackage{amssymb}%extended symbol collection 
\usepackage{amsmath}%misc. enhancements for math environments 
\usepackage{color}%color for fonts 
\usepackage[numbered,framed]{matlab-prettifier}%inputting code 
\usepackage{tcolorbox}%new problem for boxes
\usepackage{mathrsfs}%script font
\usepackage{mathabx}%mathematical fonts 
\usepackage[normalem]{ulem}%proofreading tools - strike through
\usepackage{bm}%bold math symbols 
\usepackage{centernot}%NOT for math symbols
\usepackage{enumitem}%enumerate lists
\usepackage{tikz}%plots 
\usepackage{graphicx}% key values interface for graphics
\usepackage[color,curve]{xy}%drawing diagrams
\usepackage{pgfplots}%plots for functions 
\pgfplotsset{compat=1.16}
\usepackage{pgfkeys}%keys for pgf plots
\usepackage{hyperref} %might need to take this out for import?
\usepackage{import}

\begin{comment}
% bibliography 
\usepackage{cite}

%useful packages for long documents
\usepackage{hyperref}%links contents in table of contents 
\usepackage{multicol}% allows for multicolumns in text environment 

%useful for graphs 
\usetikzlibrary{arrows}% necessary for some graphs
\usetikzlibrary{decorations.markings}%necessary for the graphs

%useful packages for images
\usepackage{caption}
\usepackage{subcaption}
\usepackage{rotating}%rotates boxed 
\end{comment}

%%%%%%%%%%%%%%%%%%%%%%%%%%%%%%%%%%%%%%%%%%%%%%%%%%%%%%
\newtheorem{theorem}{Theorem}
\newcommand{\nprob}{\medskip\hrule\medskip}
\newcommand{\tbox}[2]{\begin{tcolorbox}[colback=white, sharp corners]\textbf{\S#1.} #2 \end{tcolorbox}}
\newcommand{\enum}[1]{\begin{enumerate}[label=\alph*)] #1 \end{enumerate}}

\newcommand{\siga}{$\sigma$-algebra }
\newcommand{\st}{\text{ such that }}

\newcommand{\lr}[1]{\left(#1\right)}

\DeclareMathOperator{\imag}{Im}
\DeclareMathOperator{\real}{Re}

\begin{document}

\tbox{2.11}{ 
    For $A \in \mathcal{A}$, show $P(A) = 1 - P(A^c)$.
}
Note that by construction that for $A \subset \Omega$ that $A \bigcup A^c= \Omega$, and by additivity of a probability measure we know that $P(A\bigcup A^c)=P(A)+P(A^c)$ since $A \bigcap A^c= \emptyset$. 
But $P(A\bigcap A^c) = P(\Omega) = 1$, and thus $1 = P(A) + P(A^c)$, meaning that $P(A^c) = 1 - P(A)$.

\tbox{2.12}{
For $A,B \in \mathcal{A}$, show $P(A \cap B^c) = P(A)-P(A \cap B)$.
}
First note that
$$
    A = A \bigcap \Omega =A\bigcap (B\bigcup B^c)=(A\bigcap B) \bigcup (A\bigcap B^c).
$$ 
Now, note that $(A\bigcap B^c) \bigcap (A\bigcap B) = \emptyset$. 
Then by additivity of probability measure $P$ we have that 
$$
P(A)=P((A\bigcap B) \bigcup (A \bigcap B^c))=P(A \bigcap B) + P( A\bigcap B^c)
$$
and thus 
$$
P(A \bigcap B^c) = P(A)-P(A \bigcap B).
$$

\end{document}
