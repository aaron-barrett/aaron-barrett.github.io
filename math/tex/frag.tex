\input{header_hw.tex}

\begin{document}

\tbox{2.11}{ 
    For $A \in \mathcal{A}$, show $P(A) = 1 - P(A^c)$.
}
Note that by construction that for $A \subset \Omega$ that $A \bigcup A^c= \Omega$, and by additivity of a probability measure we know that $P(A\bigcup A^c)=P(A)+P(A^c)$ since $A \bigcap A^c= \emptyset$. 
But $P(A\bigcap A^c) = P(\Omega) = 1$, and thus $1 = P(A) + P(A^c)$, meaning that $P(A^c) = 1 - P(A)$.

\tbox{2.12}{
For $A,B \in \mathcal{A}$, show $P(A \cap B^c) = P(A)-P(A \cap B)$.
}
First note that
$$
    A = A \bigcap \Omega =A\bigcap (B\bigcup B^c)=(A\bigcap B) \bigcup (A\bigcap B^c).
$$ 
Now, note that $(A\bigcap B^c) \bigcap (A\bigcap B) = \emptyset$. 
Then by additivity of probability measure $P$ we have that 
$$
P(A)=P((A\bigcap B) \bigcup (A \bigcap B^c))=P(A \bigcap B) + P( A\bigcap B^c)
$$
and thus 
$$
P(A \bigcap B^c) = P(A)-P(A \bigcap B).
$$

\end{document}
