\documentclass[11pt,reqno]{amsart}
\usepackage[document]{ragged2e}
\usepackage{tgadventor}         %font
\usepackage{fullpage}           % sets page margins
\usepackage{amsthm}             %theorem environments
\usepackage{amssymb}            %extended symbol collection 
\usepackage{amsmath}            %misc. enhancements for math environments 
\usepackage{color}              %color for fonts 
\usepackage{tcolorbox}          %new problem for boxes
\usepackage{mathrsfs}           %script font
\usepackage{mathabx}            %mathematical fonts 
\usepackage{bm}                 %bold math symbols 
\usepackage{centernot}          %NOT for math symbols
\usepackage{enumitem}           %enumerate lists
\usepackage{tikz}               %plots 
\usepackage{graphicx}           % key values interface for graphics
\usepackage[color,curve]{xy}    %drawing diagrams
\usepackage{pgfplots}           %plots for functions 
\pgfplotsset{compat=1.16}
\usepackage{pgfkeys}            %keys for pgf plots
\usepackage{hyperref}           %might need to take this out for import?

\begin{comment}
\usepackage{cite}
\usepackage{hyperref}%links contents in table of contents 
\usepackage{caption}
\usepackage{subcaption}
\usepackage{rotating}%rotates boxed 
\end{comment}

%%%%%%%%%%%%%%%%%%%%%%%%%%%%%%%%%%%%%%%%%%%%%%%%%%%%%%
\newtheorem{theorem}{Theorem}
\newcommand{\tbox}[2]{\begin{tcolorbox}[colback=white, sharp corners]\textbf{\S#1.} #2 \end{tcolorbox}}
\newcommand{\enum}[1]{\begin{enumerate}[label=\alph*)] #1 \end{enumerate}}
\newcommand{\lr}[1]{\left(#1\right)}
\newcommand{\lrb}[1]{\left[#1\right]}
\newcommand{\bmat}[1]{\begin{bmatrix} #1 \end{bmatrix}}

\DeclareMathOperator{\imag}{Im}
\DeclareMathOperator{\real}{Re}

\begin{document}
The singular value decomposition (SVD) is fundamental to Numerical Linear Algebra as a step in many applications, such as Principal Component Analysis, and in theoretical linear algebra since its easy to manipulate with unitary invariant norms. 
There are one if not several constructive ways to prove its existence, but here I want to revisit a non-constructive proof method that is found in Trefethen and Bau.
I don't see this method used in other texts, namely Golub Van Loan and BLANK, but there's some standard tricks here that I think warrant a closer look.
% Additionally, this was my first introduction to numerical linear algebra, and I've always found this volume compelling in how manageable the lessons were with respect to how much material is presented. 
Below is my annotated proof which I hope clarifies a step or two that is not quite as clear to a modern reader. 

\end{document}
